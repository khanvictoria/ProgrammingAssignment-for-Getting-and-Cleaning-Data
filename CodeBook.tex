% Options for packages loaded elsewhere
\PassOptionsToPackage{unicode}{hyperref}
\PassOptionsToPackage{hyphens}{url}
%
\documentclass[
]{article}
\usepackage{lmodern}
\usepackage{amssymb,amsmath}
\usepackage{ifxetex,ifluatex}
\ifnum 0\ifxetex 1\fi\ifluatex 1\fi=0 % if pdftex
  \usepackage[T1]{fontenc}
  \usepackage[utf8]{inputenc}
  \usepackage{textcomp} % provide euro and other symbols
\else % if luatex or xetex
  \usepackage{unicode-math}
  \defaultfontfeatures{Scale=MatchLowercase}
  \defaultfontfeatures[\rmfamily]{Ligatures=TeX,Scale=1}
\fi
% Use upquote if available, for straight quotes in verbatim environments
\IfFileExists{upquote.sty}{\usepackage{upquote}}{}
\IfFileExists{microtype.sty}{% use microtype if available
  \usepackage[]{microtype}
  \UseMicrotypeSet[protrusion]{basicmath} % disable protrusion for tt fonts
}{}
\makeatletter
\@ifundefined{KOMAClassName}{% if non-KOMA class
  \IfFileExists{parskip.sty}{%
    \usepackage{parskip}
  }{% else
    \setlength{\parindent}{0pt}
    \setlength{\parskip}{6pt plus 2pt minus 1pt}}
}{% if KOMA class
  \KOMAoptions{parskip=half}}
\makeatother
\usepackage{xcolor}
\IfFileExists{xurl.sty}{\usepackage{xurl}}{} % add URL line breaks if available
\IfFileExists{bookmark.sty}{\usepackage{bookmark}}{\usepackage{hyperref}}
\hypersetup{
  pdftitle={CodeBook},
  hidelinks,
  pdfcreator={LaTeX via pandoc}}
\urlstyle{same} % disable monospaced font for URLs
\usepackage[margin=1in]{geometry}
\usepackage{graphicx,grffile}
\makeatletter
\def\maxwidth{\ifdim\Gin@nat@width>\linewidth\linewidth\else\Gin@nat@width\fi}
\def\maxheight{\ifdim\Gin@nat@height>\textheight\textheight\else\Gin@nat@height\fi}
\makeatother
% Scale images if necessary, so that they will not overflow the page
% margins by default, and it is still possible to overwrite the defaults
% using explicit options in \includegraphics[width, height, ...]{}
\setkeys{Gin}{width=\maxwidth,height=\maxheight,keepaspectratio}
% Set default figure placement to htbp
\makeatletter
\def\fps@figure{htbp}
\makeatother
\setlength{\emergencystretch}{3em} % prevent overfull lines
\providecommand{\tightlist}{%
  \setlength{\itemsep}{0pt}\setlength{\parskip}{0pt}}
\setcounter{secnumdepth}{-\maxdimen} % remove section numbering
% https://github.com/rstudio/rmarkdown/issues/337
\let\rmarkdownfootnote\footnote%
\def\footnote{\protect\rmarkdownfootnote}

% https://github.com/rstudio/rmarkdown/pull/252
\usepackage{titling}
\setlength{\droptitle}{-2em}

\pretitle{\vspace{\droptitle}\centering\huge}
\posttitle{\par}

\preauthor{\centering\large\emph}
\postauthor{\par}

\predate{\centering\large\emph}
\postdate{\par}

\title{CodeBook}
\date{}

\begin{document}
\maketitle

\hypertarget{i.-loading-the-data-into-r}{%
\paragraph{I. Loading the data into
R}\label{i.-loading-the-data-into-r}}

1.1 Load three datasets on test data: subject\_test.txt, X\_test.txt,
y\_test.txt from UCI HAR Dataset.\\
1.2 Name them subject\_test, X\_test and y\_test correspondingly.\\
1.3 Load three datasets on train data: subject\_train.txt, X\_train.txt,
y\_train.txt from UCI HAR Dataset.\\
1.4 Name them subject\_train, X\_train and y\_train correspondingly.

\begin{verbatim}
##               N of observations N of variables
## subject_test               2947              1
## subject_train              7352              1
## X_test                     2947            561
## X_train                    7352            561
## Y_test                     2947              1
## Y_train                    7352              1
\end{verbatim}

\hypertarget{ii.-merges-the-training-and-the-test-sets-to-create-one-data-set.}{%
\paragraph{II. Merges the training and the test sets to create one data
set.}\label{ii.-merges-the-training-and-the-test-sets-to-create-one-data-set.}}

\hypertarget{extracts-only-the-measurements-on-the-mean-and-standard-deviation-for-each-measurement.}{%
\paragraph{Extracts only the measurements on the mean and standard
deviation for each
measurement.}\label{extracts-only-the-measurements-on-the-mean-and-standard-deviation-for-each-measurement.}}

2.1 Read features.txt data into R\\
2.2 Rename columns of X\_test and X\_train according to the definitions
in features dataset\\
2.3 Merge all six files into one dataset calling it ``Activities''\\
2.4 Subset only those columns that have either ``{[}Mm{]}ean'' or
``{[}Ss{]}td'' in their names

\begin{verbatim}
##                   N of observations N of variables
## features                        561              2
## Activities                    10299            563
## subset_activities             10299             88
\end{verbatim}

\hypertarget{iii.-uses-descriptive-activity-names-to-name-the-activities-in-the-data-set}{%
\paragraph{III. Uses descriptive activity names to name the activities
in the data
set}\label{iii.-uses-descriptive-activity-names-to-name-the-activities-in-the-data-set}}

3.1 Assign appropriate labels to activities as indicated in
activity\_label.txt\\
3.2 The labels are 1 WALKING 2 WALKING\_UPSTAIRS 3 WALKING\_DOWNSTAIRS 4
SITTING 5 STANDING 6 LAYING

\begin{verbatim}
## 
##            WALKING   WALKING_UPSTAIRS WALKING_DOWNSTAIRS 
##               1722               1544               1406 
##            SITTING           STANDING             LAYING 
##               1777               1906               1944
\end{verbatim}

\hypertarget{iv.-appropriately-labels-the-data-set-with-descriptive-variable-names.}{%
\paragraph{IV. Appropriately labels the data set with descriptive
variable
names.}\label{iv.-appropriately-labels-the-data-set-with-descriptive-variable-names.}}

4.1 Rename V1 and V1.1 to ``Subject\_id'' and ``Activity''\\
4.2 Change the abbreviations ``Acc'', ``Gyro'', ``Mag'', ``t'' and ``f''
into corresponding words

\begin{verbatim}
##  [1] "Subject_id"                                            
##  [2] "Activity"                                              
##  [3] "TimeBodyAccelerometer-mean()-X"                        
##  [4] "TimeBodyAccelerometer-mean()-Y"                        
##  [5] "TimeBodyAccelerometer-mean()-Z"                        
##  [6] "TimeBodyAccelerometer-std()-X"                         
##  [7] "TimeBodyAccelerometer-std()-Y"                         
##  [8] "TimeBodyAccelerometer-std()-Z"                         
##  [9] "TimeGravityAccelerometer-mean()-X"                     
## [10] "TimeGravityAccelerometer-mean()-Y"                     
## [11] "TimeGravityAccelerometer-mean()-Z"                     
## [12] "TimeGravityAccelerometer-std()-X"                      
## [13] "TimeGravityAccelerometer-std()-Y"                      
## [14] "TimeGravityAccelerometer-std()-Z"                      
## [15] "TimeBodyAccelerometerJerk-mean()-X"                    
## [16] "TimeBodyAccelerometerJerk-mean()-Y"                    
## [17] "TimeBodyAccelerometerJerk-mean()-Z"                    
## [18] "TimeBodyAccelerometerJerk-std()-X"                     
## [19] "TimeBodyAccelerometerJerk-std()-Y"                     
## [20] "TimeBodyAccelerometerJerk-std()-Z"                     
## [21] "TimeBodyGyroscope-mean()-X"                            
## [22] "TimeBodyGyroscope-mean()-Y"                            
## [23] "TimeBodyGyroscope-mean()-Z"                            
## [24] "TimeBodyGyroscope-std()-X"                             
## [25] "TimeBodyGyroscope-std()-Y"                             
## [26] "TimeBodyGyroscope-std()-Z"                             
## [27] "TimeBodyGyroscopeJerk-mean()-X"                        
## [28] "TimeBodyGyroscopeJerk-mean()-Y"                        
## [29] "TimeBodyGyroscopeJerk-mean()-Z"                        
## [30] "TimeBodyGyroscopeJerk-std()-X"                         
## [31] "TimeBodyGyroscopeJerk-std()-Y"                         
## [32] "TimeBodyGyroscopeJerk-std()-Z"                         
## [33] "TimeBodyAccelerometerMagnitude-mean()"                 
## [34] "TimeBodyAccelerometerMagnitude-std()"                  
## [35] "TimeGravityAccelerometerMagnitude-mean()"              
## [36] "TimeGravityAccelerometerMagnitude-std()"               
## [37] "TimeBodyAccelerometerJerkMagnitude-mean()"             
## [38] "TimeBodyAccelerometerJerkMagnitude-std()"              
## [39] "TimeBodyGyroscopeMagnitude-mean()"                     
## [40] "TimeBodyGyroscopeMagnitude-std()"                      
## [41] "TimeBodyGyroscopeJerkMagnitude-mean()"                 
## [42] "TimeBodyGyroscopeJerkMagnitude-std()"                  
## [43] "FrequencyBodyAccelerometer-mean()-X"                   
## [44] "FrequencyBodyAccelerometer-mean()-Y"                   
## [45] "FrequencyBodyAccelerometer-mean()-Z"                   
## [46] "FrequencyBodyAccelerometer-std()-X"                    
## [47] "FrequencyBodyAccelerometer-std()-Y"                    
## [48] "FrequencyBodyAccelerometer-std()-Z"                    
## [49] "FrequencyBodyAccelerometer-meanFreq()-X"               
## [50] "FrequencyBodyAccelerometer-meanFreq()-Y"               
## [51] "FrequencyBodyAccelerometer-meanFreq()-Z"               
## [52] "FrequencyBodyAccelerometerJerk-mean()-X"               
## [53] "FrequencyBodyAccelerometerJerk-mean()-Y"               
## [54] "FrequencyBodyAccelerometerJerk-mean()-Z"               
## [55] "FrequencyBodyAccelerometerJerk-std()-X"                
## [56] "FrequencyBodyAccelerometerJerk-std()-Y"                
## [57] "FrequencyBodyAccelerometerJerk-std()-Z"                
## [58] "FrequencyBodyAccelerometerJerk-meanFreq()-X"           
## [59] "FrequencyBodyAccelerometerJerk-meanFreq()-Y"           
## [60] "FrequencyBodyAccelerometerJerk-meanFreq()-Z"           
## [61] "FrequencyBodyGyroscope-mean()-X"                       
## [62] "FrequencyBodyGyroscope-mean()-Y"                       
## [63] "FrequencyBodyGyroscope-mean()-Z"                       
## [64] "FrequencyBodyGyroscope-std()-X"                        
## [65] "FrequencyBodyGyroscope-std()-Y"                        
## [66] "FrequencyBodyGyroscope-std()-Z"                        
## [67] "FrequencyBodyGyroscope-meanFreq()-X"                   
## [68] "FrequencyBodyGyroscope-meanFreq()-Y"                   
## [69] "FrequencyBodyGyroscope-meanFreq()-Z"                   
## [70] "FrequencyBodyAccelerometerMagnitude-mean()"            
## [71] "FrequencyBodyAccelerometerMagnitude-std()"             
## [72] "FrequencyBodyAccelerometerMagnitude-meanFreq()"        
## [73] "FrequencyBodyBodyAccelerometerJerkMagnitude-mean()"    
## [74] "FrequencyBodyBodyAccelerometerJerkMagnitude-std()"     
## [75] "FrequencyBodyBodyAccelerometerJerkMagnitude-meanFreq()"
## [76] "FrequencyBodyBodyGyroscopeMagnitude-mean()"            
## [77] "FrequencyBodyBodyGyroscopeMagnitude-std()"             
## [78] "FrequencyBodyBodyGyroscopeMagnitude-meanFreq()"        
## [79] "FrequencyBodyBodyGyroscopeJerkMagnitude-mean()"        
## [80] "FrequencyBodyBodyGyroscopeJerkMagnitude-std()"         
## [81] "FrequencyBodyBodyGyroscopeJerkMagnitude-meanFreq()"    
## [82] "angle(TimeBodyAccelerometerMean,gravity)"              
## [83] "angle(TimeBodyAccelerometerJerkMean),gravityMean)"     
## [84] "angle(TimeBodyGyroscopeMean,gravityMean)"              
## [85] "angle(TimeBodyGyroscopeJerkMean,gravityMean)"          
## [86] "angle(X,gravityMean)"                                  
## [87] "angle(Y,gravityMean)"                                  
## [88] "angle(Z,gravityMean)"
\end{verbatim}

\hypertarget{v.-from-the-data-set-in-step-4-creates-a-second-independent-tidy-data-set-with-the-average-of-each-variable-for-each-activity-and-each-subject.}{%
\paragraph{V. From the data set in step 4, creates a second, independent
tidy data set with the average of each variable for each activity and
each
subject.}\label{v.-from-the-data-set-in-step-4-creates-a-second-independent-tidy-data-set-with-the-average-of-each-variable-for-each-activity-and-each-subject.}}

5.1 Melt the subset\_activities dataset by Subject\_id and Activity\\
5.2 Group by Subject\_id, Activity, variables (mean or std) of the
feature vector and find average across every variables, activity and
subject\\
5.3 Save as ``tidy\_data.txt'' and push to the repo

\begin{verbatim}
##   Activity Subject_id                   Measure_Axis     Value
## 1 STANDING          2 TimeBodyAccelerometer-mean()-X 0.2571778
## 2 STANDING          2 TimeBodyAccelerometer-mean()-X 0.2860267
## 3 STANDING          2 TimeBodyAccelerometer-mean()-X 0.2754848
## 4 STANDING          2 TimeBodyAccelerometer-mean()-X 0.2702982
## 5 STANDING          2 TimeBodyAccelerometer-mean()-X 0.2748330
\end{verbatim}

\begin{verbatim}
##                Activity Subject_id         Measure_Axis      Value
## 885710 WALKING_UPSTAIRS         30 angle(Z,gravityMean) 0.04981914
## 885711 WALKING_UPSTAIRS         30 angle(Z,gravityMean) 0.05005256
## 885712 WALKING_UPSTAIRS         30 angle(Z,gravityMean) 0.04081119
## 885713 WALKING_UPSTAIRS         30 angle(Z,gravityMean) 0.02533948
## 885714 WALKING_UPSTAIRS         30 angle(Z,gravityMean) 0.03669484
\end{verbatim}

\end{document}
